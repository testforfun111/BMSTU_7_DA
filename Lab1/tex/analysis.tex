\chapter{Аналитическая часть}
В этом разделе будут рассмотрены классический алгоритм работы шифровальной машины <<Энигма>>, а также её вариант, использованный во время Второй мировой войны.

Шифровальная машина <<Энигма>> состоит из следующих деталей: роторы, входное колесо, рефлектор, а также коммутационная панель.

\section{Роторы}

<<Энигма>> предназначена для шифрации сообщений, написанных на английском языке.
Ротор --- прикреплённый к шестерёнке с 26 зубцами (по одному на каждую букву алфавита) элемент, предназначенный для преобразования одной буквы в другую.

В разное время в разных реализациях <<Энигмы>> использовалось разное количество роторов. Во время Второй мировой войны использовались 3 ротора, причём всего было 10 роторов, преобразовывающих буквы в соответствии с таблицей \ref{tbl:rotors}.

\begin{table}[h]
\small
\setlength{\tabcolsep}{3pt}
	\begin{center}
		\begin{threeparttable}
		\captionsetup{justification=raggedright,singlelinecheck=off}
		\caption{\label{tbl:rotors} Преобразования роторов <<Энигмы>>}
		\begin{tabular}{|c|c|c|c|c|c|c|c|c|c|c|c|c|c|c|c|c|c|c|c|c|c|c|c|c|c|c|}
			\hline
			Ротор & A & B & C & D & E & F & G & H & I & J & K & L & M & N & O & P & Q & R & S & T & U & V & W & X & Y & Z \\
			\hline
			I & E & K & M & F & L & G & D & Q & V & Z & N & T & O & W & Y & H & X & U & S & P & A & I & B & R & C & J \\
			\hline
			II & A & J & D & K & S & I & R & U & X & B & L & H & W & T & M & C & Q & G & Z & N & P & Y & F & V & O & E \\
			\hline
			III & B & D & F & H & J & L & C & P & R & T & X & V & Z & N & Y & E & I & W & G & A & K & M & U & S & Q & O \\
			\hline
			IV & E & S & O & V & P & Z & J & A & Y & Q & U & I & R & H & X & L & N & F & T & G & K & D & C & M & W & B \\
			\hline
			V& V & Z & B & R & G & I & T & Y & U & P & S & D & N & H & L & X & A & W & M & J & Q & O & F & E & C & K \\
			\hline
			VI & J & P & G & V & O & U & M & F & Y & Q & B & E & N & H & Z & R & D & K & A & S & X & L & I & C & T & W \\
			\hline
			VII & N & Z & J & H & G & R & C & X & M & Y & S & W & B & O & U & F & A & I & V & L & P & E & K & Q & D & T \\
			\hline
			VIII & F & K & Q & H & T & L & X & O & C & B & J & S & P & D & Z & R & A & M & E & W & N & I & U & Y & G & V \\
			\hline
			IX & L & E & Y & J & V & C & N & I & X & W & P & B & Q & M & D & R & T & A & K & Z & G & F & U & H & O & S \\
			\hline
			X & F & S & O & K & A & N & U & E & R & H & M & B & T & I & Y & C & W & L & Q & P & Z & X & V & G & J & D \\
			\hline
		\end{tabular}
		\end{threeparttable}
	\end{center}
	
\end{table}

\section{Входное колесо}

Входное колесо --- элемент, позволяющий выставить роторы в необходимые значения. 
В физической машине было 3 отверстия, позволяющих просматривать, в каком состоянии находится каждый ротор.
Положения роторов является ключевым для процесса шифрования, поскольку в зависимости от них одно и то же сообщение будет зашифровано по-разному и будет требовать сответствующих начальных значений роторов для дешифрации.

\section{Рефлектор}

Рефлектор --- элемент, попарно соединяющий контакты последнего ротора, тем самым направляя ток обратно на последний ротор. Так, после этого электрический сигнал пойдёт в обратном направлении, пройдя через все роторы повторно. 
Во время Второй мировой войны было создано 2 рефлектора, представленных в таблице

\begin{table}[h]
\small
\setlength{\tabcolsep}{3pt}
	\begin{center}
		\begin{threeparttable}
		\captionsetup{justification=raggedright,singlelinecheck=off}
		\caption{\label{tbl:reflectors} Преобразования роторов <<Энигмы>>}
		\begin{tabular}{|c|c|c|c|c|c|c|c|c|c|c|c|c|c|c|c|c|c|c|c|c|c|c|c|c|c|c|}
			\hline
			Рефлектор & A & B & C & D & E & F & G & H & I & J & K & L & M & N & O & P & Q & R & S & T & U & V & W & X & Y & Z \\
			\hline
			I & F & V & P & J & I & A & O & Y & E & D & R & Z & X & W & G & C & T & K & U & Q & S & B & N & M & H & L \\
			\hline
			II & Y & R & U & H & Q & S & L & D & P & X & N & G & O & K & M & I & E & B & F & Z & C & W & V & J & A & T \\
			\hline
		\end{tabular}
		\end{threeparttable}
	\end{center}
	
\end{table}

\section{Коммутационная панель}

Коммутационная панель позволяет оператору шифровальной машины варьировать содержимое проводов, попарно соединяющих буквы английского алфавита. 
Эффект состоял в том, чтобы усложнить работу машины, не увеличивая число роторов.
Так, если на коммутационной панели соединены буквы 'A' и 'Z', то каждая буква 'A', проходящая через коммутационную панель, будет заменена на 'Z'  и наоборот. Сигналы попадали на коммутационную панель 2 раза: в начале и в конце обработки отдельного символа.