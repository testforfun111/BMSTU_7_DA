\chapter{Технологическая часть}

В данном разделе будут рассмотрены средства реализации, а также представлены листинги реализаций алгоритма шифрования машины <<Энигма>>.

\section{Средства реализации}
В данной работе для реализации был выбран язык программирования $C$. Данный язык удоволетворяет поставленным критериям по средствам реализации.

\section{Реализация алгоритма}

В листингах \ref{lst:enigma1} представлена реализация алгоритма шифрования машины <<Энигма>>.

\begin{center}
    \captionsetup{justification=raggedright,singlelinecheck=off}
    \begin{lstlisting}[label=lst:enigma1,caption=Реализация алгоритма шифрования машины <<Энигма>>]
#include <stdio.h>
#include <stdint.h>
#include <stdlib.h>
#include <string.h>

#define ALPHABET_SIZE 16

const int8_t ALPHABET[ALPHABET_SIZE] = {0, 1, 2, 3, 4, 5, 6, 7, 8, 9, 10, 11, 12, 13, 14, 15};

void my_memcpy(int8_t *dst, const int8_t *src, int len)
{
	for (int i = 0; i < len; i++)
	{
		dst[i] = src[i];
	}
}

int8_t find(const int8_t* nums, int8_t num, int size) {
	for (int i = 0; i < size; i++) {
		if (num == nums[i])
		return i;
	}
	return -1; 
}

// Rotor struct and methods
typedef struct {
	int8_t wiring[ALPHABET_SIZE];
} Rotor;

void initRotor(Rotor* rotor, const int8_t* wiring, int position) {
	for (int i = position; i < ALPHABET_SIZE; i++) {
		rotor->wiring[i - position] = wiring[i];
	}
	for (int i = 0; i < position; i++) {
		rotor->wiring[ALPHABET_SIZE - position + i] = wiring[i];
	}
}

int8_t rotorEncryptForward(Rotor* rotor, int8_t c) {
	int index = find(rotor->wiring, c, ALPHABET_SIZE);
	return ALPHABET[index];
}

int8_t rotorEncryptBackward(Rotor* rotor, int8_t c) {
	int index = find(ALPHABET, c, ALPHABET_SIZE);
	return rotor->wiring[index];
}

void rotorRotate(Rotor* rotor) {
	int8_t tmp = rotor->wiring[0];
	for (int i = 0; i < ALPHABET_SIZE - 1; i++) {
		rotor->wiring[i] = rotor->wiring[i + 1];
	}
	rotor->wiring[ALPHABET_SIZE - 1] = tmp;
}

// Commutation Panel struct and methods
typedef struct {
	int8_t wiring[ALPHABET_SIZE];
	int8_t originalWiring[ALPHABET_SIZE];
} CommutationPanel;

void initCommutationPanel(CommutationPanel* panel, const int8_t* wiring) {
	my_memcpy(panel->originalWiring, wiring, ALPHABET_SIZE);
	my_memcpy(panel->wiring, wiring, ALPHABET_SIZE);
	for (int i = 0; i < ALPHABET_SIZE; i += 2) {
		int8_t temp = panel->wiring[i];
		panel->wiring[i] = panel->wiring[i + 1];
		panel->wiring[i + 1] = temp;
	}
}

int8_t commutate(CommutationPanel* panel, int8_t c) {
	int idx = find(panel->originalWiring, c, ALPHABET_SIZE);
	if (idx != -1) {
		return panel->wiring[idx];
	}
	return c; // No change if not found
}

// Reflector struct and methods
typedef struct {
	int8_t wiring[ALPHABET_SIZE];
} Reflector;

void initReflector(Reflector* reflector, const int8_t* wiring) {
	my_memcpy(reflector->wiring, wiring, ALPHABET_SIZE);
	for (int i = 0; i < ALPHABET_SIZE; i += 2) {
		int8_t temp = reflector->wiring[i];
		reflector->wiring[i] = reflector->wiring[i + 1];
		reflector->wiring[i + 1] = temp;
	}
}

int8_t reflect(Reflector* reflector, int8_t c) {
	return reflector->wiring[find(ALPHABET, c, ALPHABET_SIZE)];
}

// Enigma Machine structure
typedef struct {
	Rotor* rotors;
	int rotorCount;
	Reflector reflector;
	CommutationPanel panel;
	int cnt;
} Enigma;

void initEnigma(Enigma* enigma, Rotor* rotors, int rotorCount, Reflector reflector, CommutationPanel panel) {
	enigma->rotors = rotors;
	enigma->rotorCount = rotorCount;
	enigma->reflector = reflector;
	enigma->panel = panel;
	enigma->cnt = 0;
}

void rotateRotors(Enigma* enigma) {
	enigma->cnt++;
	rotorRotate(&enigma->rotors[0]);
	if (enigma->cnt % ALPHABET_SIZE == 0) {
		rotorRotate(&enigma->rotors[1]);
	}
	if (enigma->cnt % (ALPHABET_SIZE * ALPHABET_SIZE) == 0) {
		rotorRotate(&enigma->rotors[2]);
	}
}

int8_t encryptChar(Enigma* enigma, int8_t p1, int8_t p2) {
	// Process first part of the byte
	p1 = commutate(&enigma->panel, p1);
	for (int i = 0; i < enigma->rotorCount; i++) {
		p1 = rotorEncryptForward(&enigma->rotors[i], p1);
	}
	p1 = reflect(&enigma->reflector, p1);
	for (int i = enigma->rotorCount - 1; i >= 0; i--) {
		p1 = rotorEncryptBackward(&enigma->rotors[i], p1);
	}
	p1 = commutate(&enigma->panel, p1);
	
	// Process second part of the byte
	p2 = commutate(&enigma->panel, p2);
	for (int i = 0; i < enigma->rotorCount; i++) {
		p2 = rotorEncryptForward(&enigma->rotors[i], p2);
	}
	p2 = reflect(&enigma->reflector, p2);
	for (int i = enigma->rotorCount - 1; i >= 0; i--) {
		p2 = rotorEncryptBackward(&enigma->rotors[i], p2);
	}
	p2 = commutate(&enigma->panel, p2);
	
	// Combine both parts into one byte
	return (p1 << 4) | p2;
}

char* encryptMessage(Enigma* enigma, const char* message) {
	int len = strlen(message);
	char* encryptedMessage = malloc((len + 1) * sizeof(char));  
	for (int i = 0; i < len; i++) {
		int8_t p1 = (message[i] & 0xf0) >> 4;
		int8_t p2 = (message[i] & 0x0f);
		encryptedMessage[i] = encryptChar(enigma, p1, p2);
		rotateRotors(enigma);
	}
	encryptedMessage[len] = '\0'; 
	return encryptedMessage;
}

void encryptFile(Enigma* enigma, const char* inputFile, const char* outputFile) {
	FILE* inFile = fopen(inputFile, "rb");
	FILE* outFile = fopen(outputFile, "wb");
	
	if (!inFile || !outFile) {
		printf("Cannot open files!\n");
		return;
	}
	
	int8_t buffer;
	while (fread(&buffer, sizeof(int8_t), 1, inFile)) {
		int8_t p1 = (buffer & 0xf0) >> 4;
		int8_t p2 = buffer & 0x0f;
		buffer = encryptChar(enigma, p1, p2);
		rotateRotors(enigma);
		fwrite(&buffer, sizeof(int8_t), 1, outFile);
	}
	
	fclose(inFile);
	fclose(outFile);
}

\end{lstlisting}
\end{center}

\section{Тестирование}
\begin{table}[ht!]
	\begin{center}
		\captionsetup{justification=raggedright,singlelinecheck=off}
		\caption{\label{tbl:functional_test} Функциональные тесты}
		\begin{tabular}{|c|c|c|}
			\hline
			Входная строка & Выходная строка \\ 
			\hline
			Hello world & >------- \\
			-------  & Hello world\\
			
			\hline
		\end{tabular}
	\end{center}
\end{table}

\section*{Вывод}

Были представлены листинги реализаций алгоритма шифрования в машине <<Энигма>> согласно алгоритму, представленному в первой части, а также проведено тестирование разработанной программы.